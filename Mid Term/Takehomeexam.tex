% LaTeX Article Template - customizing header and footer
\documentclass{article}

\newtheorem{thm}{Theorem}

% Set left margin - The default is 1 inch, so the following 
% command sets a 1.25-inch left margin.
\setlength{\oddsidemargin}{0.25in}

% Set width of the text - What is left will be the right margin.
% In this case, right margin is 8.5in - 1.25in - 6in = 1.25in.
\setlength{\textwidth}{6in}

% Set top margin - The default is 1 inch, so the following 
% command sets a 0.75-inch top margin.
\setlength{\topmargin}{-0.25in}

% Set height of the header
\setlength{\headheight}{0.3in}

% Set vertical distance between the header and the text
\setlength{\headsep}{0.2in}

% Set height of the text
\setlength{\textheight}{9in}

% Set vertical distance between the text and the
% bottom of footer
\setlength{\footskip}{0.1in}

% Set the beginning of a LaTeX document
\usepackage{multirow}
\usepackage{fullpage}
\usepackage{graphicx}
\usepackage{amsthm}
\usepackage{amssymb}
\usepackage{url}
\usepackage{algpseudocode}
\graphicspath{%
    {converted_graphics/}% inserted by PCTeX
    {/}% inserted by PCTeX
}
%%%%%%%%%%%%%%%%%%%%%%%%%%%%%




\begin{document}\title{Midterm\\ Computer Science \\ Fall 2015\\ B565}         % Enter your title between curly braces
\author{Ganesh Nagarajan\\gnagaraj@indiana.edu\\ Sunday, October 25, 9:00 p.m.}        % Enter your name between curly braces
\date{\today}          % Enter your date or \today between curly braces
\maketitle


% Redefine "plain" pagestyle
\makeatother     % `@' is restored as a "non-letter" character



% Set to use the "plain" pagestyle
\pagestyle{plain}
\section*{Solutions}
All the work herein is solely mine.\\
Inorder to solve the given problem, following were the libraries and tools that contributed significantly.

\begin{tabular}{|c|c|} \hline
$Tool/Library$ & $Version$ \\ \hline \hline 
R & 3.2.2 \\
\hline

\section{Application [250pts]}

Assume you work for \texttt{Filmflix.com}, an online movie rental business.  The data from \texttt{Filmflix.com} is found in Table\,\ref{t1}.   Read the dialogue and answer appropriately.

\begin{enumerate} \item The client says to you, ``We'd like to understand our customers broadly from their tastes in movies.  What do you suggest?"  \\ \noindent ``I'll do a $k$-means and show you the results!  
\item The client says to you, ``I was reading about association rules.  These rules could help us promote certain movies.  What does the data suggest?"\\ \noindent ``Well, we can do either genre or movies--or even both.  Why don't I generate a couple of rules for each individually and suggest some promotions."
\item The client says to you, ``My favorite movies are 2,5,8, and 9."  I'm curious, what customers is nearest to me in my choices?"\\ \noindent ``I suggest a $knn$.  I'll find out!"
\item The client says to you, ``We have a recommendation process--we ask the genres you like and we suggest the movie; but, it's not very scientific.  If I like {\it Action} and {\it Drama} what are the three best recommendations for movies?" \\ \noindent ``We can use a N\"{a}ive Bayes for this.  I'll let you know."
\item The client says to you, ``Right now, the genres are kind of unrelated to one another.  I wonder if you could build a tree that shows how they are related from viewers' points of view?" \\ \noindent ``I'll do aglommerative clustering, and we'll see if we can interpret the tree."
\end{enumerate}


\begin{table}\label{t1}
\begin{minipage}[c]{5cm}
\begin{tabular}{ll}
\multicolumn{2}{c}{\textsf{GC}}\\ \hline \hline
\textsf{Genre} & \textsf{Code} \\ \hline
Romance & r \\
Science Fiction & s \\
Horror & h \\
Comedy & c \\
 Drama & d \\
Action & a \\
Documentary & o \\
Classic & l \\
\end{tabular}
\end{minipage}
\begin{minipage}[c]{3cm}
\begin{tabular}{cl}
\multicolumn{2}{c}{\textsf{MIDG}}\\ \hline \hline 
\textsf{Movie ID} & \textsf{Genre} \\ \hline
1 & r,s \\
2 & o,l,a\\
3 & c,d, h\\
4 & s, l, o, a\\
5 & a, d, r\\
6 & d, h, c\\
7 & a, d, c, o\\
8 & h, l, r\\
9 & s, d\\
10 & c, r\\
\end{tabular}
\end{minipage}
\hspace{2cm}
\begin{minipage}[c]{8cm}
\begin{tabular}{cl}
\multicolumn{2}{c}{\textsf{CIDM}} \\ \hline \hline
\textsf{Customer ID} & \textsf{Movies} \\ \hline
CID1 & 1,3,5,5,10,8 \\
CID2 & 4,1,2,3\\
CID3 & 7,8,1\\
CID4 & 2\\
CID5 & 4,8,10\\
CID6 & 3,9,10,1\\
CID7 & 1,2,3\\
CID8 & 5,4,9,5\\
CID9 & 10,1,2,23\\
CID10 & 2,4,3,7,9\\
CID11 & 1,10,8 \\
CID12 & 3,5,1,2,\\
CID13 & 8,1,7\\
CID14 & 5,2,8\\
\end{tabular}
\end{minipage}
\caption{Data from an online movie rental company.  The GC table gives the genre codes, MIDG the associated genres with the 10 movies they rent online, and CIDM the movies that a customer has rented.}
\end{table}



\section{Equivocation: $k,\ell$-means Algorithm [100pts]}
This problem asks you to modify the $k$-means algorithm to $k,\ell$.  The $\ell$ is the best number of centroids that the datum matches.  So, $4,2$-means each datum is matched to the 2 closest centroids.  You can assume that we're using {\it average} for the best representative.  Complete the algorithm below for $k,\ell$-means.  The final result {\it must} be an actual partition.

\begin{center}
\begin{algorithmic}[1]\label{k,l-means}
\State{\bf ALGORITHM} \texttt{k,l-means}
\State {\bf INPUT} (\textsf{data} $\Delta$, distance $d:\Delta^2\rightarrow \mathbb{R}_{\geq 0}$, \textsf{centoid number} $k$, \textsf{Closest Matches} $\ell$, \textsf{threshold} $\tau$)
\State {\bf OUTPUT} (\textsf{Set of centoids} $c_1, c_2, \ldots, c_k$)
\State Assume centroid is structure $c = (v \in DOM(\Delta), B\subseteq \Delta)$
\State  $c.v$ is the centroid value and $c.B$ is the set of nearest points.
\State $\tau$ is a percentage change from previous centroids
\State For example, $\{c_1, c_2, \ldots, c_k\}$ is previous and $\{d_1, d_2, \ldots, d_k\}$ is current
\State Total difference is $\Sigma_i \Sigma_j d(c_i, d_j)$
\State $Dom(\Delta)$ denotes domain of object.
\State $i = 0$
\Comment{Initialize iterate where superscript is iteration}
\For{$j = 1,k$}
\Comment{Initialize Centroids}
\State $c_j^i.v \gets  random(Dom(\Delta))$
\State $c_j^i.B \gets \emptyset$
\EndFor
\State $f_i = \Sigma_{j=1}^k\Sigma_{\ell = 1}^k d(c_j^i.v, random(Dom(\Delta))$
\Comment{Bootstrap difference between past centroids and current}
\Repeat
\State \texttt{COMPLETE CODE}
\Until{$(|f_i - f_{i-1}| < \tau(f_{i-1}))$}
\State {\bf return} ($c_1^i, c_2^i, \ldots, c_3^i$) 
\end{algorithmic}
\end{center}

Modify your $k$-means code to allow $\ell=2$ and rerun your analysis on the breast cancer data\cite{WMbreast90}.  Discuss your results with respect to your earlier results.

\section{Connections [75pts] }

\begin{table}
\begin{center}
\begin{tabular}{ll}
\multicolumn{2}{c}{\textsf{Survey}} \\ \hline
\textsf{Q1: Do you own your home?}  & \textsf{Q2: Do you own your car?} \\ \hline \hline 
Yes & No \\
No & Yes \\
Maybe & No \\
$\vdots$ & $\vdots$ \\ \hline
\end{tabular}
\end{center}
\end{table}

After examining the results of the survey, you find that there are only three kinds of responses in the same proportions: (Yes, No), (No, Yes), and (Maybe, No).  
\begin{enumerate}
\item Are the questions Q1 and Q2 statistically dependent?
\item Are the questions Q1 and Q2 statistically correlated?
\end{enumerate}
\bibliographystyle{unsrt} 
\bibliography{foo}
\end{document}
